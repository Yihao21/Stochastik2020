% header
\documentclass[10pt,a4paper]{article}

\usepackage[latin1]{inputenc}
\usepackage{hyperref}
\usepackage{amssymb}
\usepackage{enumerate}
\usepackage{amsmath}
\usepackage{ngerman}

% the document
\begin{document}

% create the title
% Please replace the data in brackets [] with actual data.
\title{Abgabe - �bungsblatt [$1$]\\
\small{Angewandte Mathematik: Stochastik}}
\author{ [Vincent Sch�nbach] \and [Yihao Wang]}
\date{\today}
\maketitle

\section*{Aufgabe 1}
\begin{enumerate}[a)]
\item Immer falsch. $w$ ist Tupel und $A$ ist Menge.
\item Immer richtig. Menge selber geh�rt zu seiner eigenen Potenzmenge.
\item Immer richtig. $w \in \Omega \Rightarrow \{ w \} \in A$.
\item Immer richtig. $w$ ist Tupel.
\item Immer falsch. $w$ ist Tupel, nicht Menge.
\item Immer falsch. Da $w_i$ kein Tupel.
\item Im Allgemeinen falsch.
\item Immer richtig. A ist die Potenzmenge von $\Omega$. Alle diese Elemente sind drin.
\end{enumerate}

\section*{Aufgabe 2}
\begin{enumerate}[a)]
 \item 
 \begin{align}
 |N^M| &\overset{Def.}{=} |\{f Abb.\,|\,f: M \rightarrow N\}| \notag \\
&= Anzahl Moeglichkeiten |M| Elemente auf |N| Elemente abzubilden. \notag \\
&= |N|^{|M|} \notag
\end{align}
 \item
$|2^M| \overset{Def.}{=} |\{ X\,|\,X \subseteq M \}|$
 
F�r jede Teilmenge $X \subseteq M$ exsitiert genau eine Funktion mit:
 \begin{align}
 f_x: M \rightarrow \{0,1\} \notag\\
 f_x(a) = \left\{
          \begin{aligned}
           &0, falls \quad a \notin X \\
           &1, sonst
          \end{aligned}\right.  \notag
 \end{align}
  Somit ist:
\begin{align}
|2^M| &= |\{ f Abb.\,|\,f: M \rightarrow \{0,1\}\}| \notag \\
 &= |\{0,1\}^M| \notag \\
 &= 2^{|M|} \notag
\end{align}

\end{enumerate}



\section*{Aufgabe 3}
\begin{enumerate}[a)]
 \item Beweis mit vollst�ndigen Induktion:
 
IA: $\quad n = 0 \Rightarrow k = 0$
 \begin{align}
  \binom{0}{0} &= |\{N \subset \{1,\ldots,0\} | |N| = 0\}| (Nach Def.)\notag\\
 &= 1 =  \frac{0!}{0! \bullet (0 - 0)!} \notag 
\end{align}

$IS: n \Rightarrow n + 1$
\begin{align}
 \binom{n+1}{k} &= |\{N \subset \{1,\ldots,n + 1\} | |N| = k\}| \notag \\
 &= |\{N \subset \{1,\ldots,n\} | |N| = k\}| \notag \\
 &+ |\{N \subset \{1,\ldots,n + 1\} | |N| = k \cap n + 1 \}| \notag \\
 &= \binom{n}{k} + |\{N \subset \{1,\ldots,n\} | |N| = k - 1\}| \notag \\
 &= \binom{n}{k} + \binom{n}{k-1} \notag \\
 &\overset{IV}{=} \frac{n!}{k! \bullet (n - k)!} + \frac{n!}{(k - 1)! \bullet (n - k + 1)!} \notag \\
 &= \frac{n!}{(k - 1)! \bullet (n - k)!} \times (\frac{1}{k} + \frac{1}{n - k + 1}) \notag \\
 &= \frac{n!}{(k - 1)! \bullet (n - k)!} \times \frac{n - k + 1 + k}{k \bullet (n - k + 1)} \notag \\
 &= \frac{(n + 1)!}{k! \bullet ((n + 1) - k)!} \notag 
 \end{align}


 \item Die Reihenfolge von Kombination eindeutig: 
 \[
 \frac{1}{49 \times 48 \times \ldots \times 44}
 \]
 
 Die Reihenfolge ist egal:
  \[
 \frac{1}{\binom{49}{6}} = \frac{1}{13983816}
 \]
\end{enumerate}

\section*{Aufgabe 4}
\begin{enumerate}[a)]
 \item 
 Z�hler: Die Wahrscheinlichkeit, die k-mal Kopf von n-mal zu erhalten.
 
 Nenner: Alle Ergebnisse $ \Rightarrow $ n-mal fair werfen $({\frac 12})^{n}$
\item
\[
 \frac{\binom{n}{1}}{2^{n}} = \frac{n}{2^{n}}
\]

 \end{enumerate}

\end{document}

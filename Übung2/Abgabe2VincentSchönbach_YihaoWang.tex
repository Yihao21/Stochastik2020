% header
\documentclass[10pt,a4paper]{article}

\usepackage[latin1]{inputenc}
\usepackage{hyperref}
\usepackage{enumerate}
\usepackage{amssymb}
\usepackage{amsmath}
\usepackage{ngerman}

% the document
\begin{document}

% create the title
% Please replace the data in brackets [] with actual data.
\title{Abgabe - �bungsblatt [$2$]\\
\small{Angewandte Mathematik: Stochastik}}
\author{ [Vincent Sch�nbach] \and [Yihao Wang]}
\date{\today}
\maketitle

\section*{Aufgabe 1}
\begin{enumerate}[a)]
 \item Ein System M von Teilmengen von $\Omega$ hei�t $\sigma - $ Algebra �ber $\Omega$, wenn gilt: $\Omega \in M$, aber es ist nicht der Fall.
 \item $\{\emptyset, \Omega, \{r,g\}, \{b\}, \{r,g,b\}, \{r,g,a\}, \{b,a\},\{a\}\}$
 \item 
 \begin{tabular}{|l|l|}
  \hline
  {\bfseries Ererignis} & {\bfseries Wahrscheinlichkeit P}\\
  \hline
  $\emptyset$ & 0\\
  \hline
  $\Omega$ & 1\\
  \hline
  $\{r,g\}$ & $3 / 8$\\
  \hline
  $\{b\}$ & $1 / 4$\\
  \hline
  $\{r,g,b\}$ & $5 / 8$\\
  \hline
  $\{r,g,a\}$ & $3 / 4$\\
  \hline
  $\{b,a\}$ & $5 / 8$\\
  \hline
  $\{a\}$ & $3 / 8$\\
  \hline
 \end{tabular}

\end{enumerate}


\section*{Aufgabe 2}
\begin{enumerate}[a)]
 \item 
 $\Omega_7 = [1:6]^7 = \{(\omega_1, \ldots, \omega_7) \mid \forall i: \omega_i \in [1:6]\}$
 \item
 $A = \{(\omega_1, \ldots, \omega_7) \in \Omega \mid \forall j \in [1:6], \exists i \in [1:7]: \omega_i = j\}$\\
 $B = \{(\omega_1, \ldots, \omega_7) \in \Omega \mid \sum_{i=1}^{7} \omega_i \quad mod \quad 2 = 0\}$
 \item
 $\vert \Omega \vert = 6^7$\\
 $\vert A \vert = 6! \times \binom{7}{1} \times 6$ Begr�ndung: Man w�hlt zun�chst einen freien Platz von 7 Pl�tze, und in diesen Platz gibt es 6 m�glich W�rfeln. 6 Fakult�t bedeutet, der Rest darf jede Zahl genau ein mal erscheinen, und die Reihenfolge davon ist auch wichtig.\\
 $\vert B \vert = \frac{6^7}{2}$ Begr�ndung: gerade+gerade = gerade, nicht gerade + nicht gerade = gerade. Die sind alle F�llen.
\end{enumerate}

\section*{Aufgabe 3}
z.zg: $\mathcal{A}_1,\mathcal{A}_2, \ldots \in \mathcal{A} \Rightarrow \bigcap\limits_{i \geq 1}\mathcal{A}_i \in  \mathcal{A}$
\begin{align}
 &Bew: Sei \quad\mathcal{A}_1,\mathcal{A}_2, \ldots \in \mathcal{A}\notag \\
 &\overset{Def (b)}{\Rightarrow} \mathcal{A}_1 ^ c,\mathcal{A}_2 ^ c, \ldots \in \mathcal{A}\notag \\
 &\overset{Def (c)}{\Rightarrow} \bigcup\limits_{i \geq 1}\mathcal{A}_i ^ c\in  \mathcal{A} \notag \\
 &\Rightarrow (\bigcap\limits_{i \geq 1}\mathcal{A}_i)^c \in \mathcal{A} (De-morgansche Gesetz)\notag \\
 &\overset{Def (b)}{\Rightarrow} \bigcap\limits_{i \geq 1}\mathcal{A}_i \in \mathcal{A} \notag
 \end{align}


\section*{Aufgabe 4}

\end{document}

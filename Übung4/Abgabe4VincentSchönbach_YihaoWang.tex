% header
\documentclass[10pt,a4paper]{article}

\usepackage[utf8]{inputenc}
\usepackage{hyperref}
\usepackage{enumerate}
\usepackage{amssymb}
\usepackage{amsmath}
\usepackage{amsthm}
\usepackage{colortbl}
\usepackage{diagbox}
\usepackage{ngerman}

% the document
\begin{document}

% create the title
% Please replace the data in brackets [] with actual data.
\title{Abgabe - Übungsblatt [$4$]\\
\small{Angewandte Mathematik: Stochastik}}
\author{ [Vincent Schönbach] \and [Yihao Wang]}
\date{\today}
\maketitle

\section*{Aufgabe 1}

\newpage
\section*{Aufgabe 2}
\begin{enumerate}[a)]
 \item 
$\Omega = [1:6]^3 = \{(\omega_1, \omega_2, \omega_3) \mid \forall i: \omega_i \in [1:6]\}$\\
$A = 2 ^ \Omega$\\
$\Omega^\prime = \{(\omega_1, \omega_2, \omega_3) \mid \forall i: \omega_i \in [1:6] \land \omega_1 \leq \omega_2 \leq \omega_3 \}$\\
$A^\prime  = 2 ^ {\Omega^\prime} $\\

$Y:(\Omega) \rightarrow (\Omega')$\\
$(\omega_1, \omega_2, \omega_3) \mapsto (min\{\omega_1, \omega_2, \omega_3 \},
min(\{\omega_1, \omega_2, \omega_3\} \setminus min\{\omega_1, \omega_2, \omega_3\}), max\{\omega_1, \omega_2, \omega_3\})

)$
\item

Verteilung $P_Y$ von $Y$:\\
\textbf{$\#$unterschiedlicher Augenzahlen 1:}\\
$P_Y(\{(a,a,a)\}) = \frac{1}{|\Omega|} \cdot 1 \cdot 1! = \frac{1}{216}, \forall a \in [1:6]$\\
\textbf{$\#$unterschiedlicher Augenzahlen 2 (2 Fälle):}\\
$P_Y(\{(a,b,b)\}) = \frac{1}{|\Omega|} \cdot 3 \cdot 2! \cdot \frac{1}{2} = \frac{3}{216}, \forall a,b \in [1:6]$\\
$P_Y(\{(a,a,b)\}) = \frac{1}{|\Omega|} \cdot 3 \cdot 2! \cdot \frac{1}{2} = \frac{3}{216}, \forall a,b \in [1:6]$\\
\textbf{$\#$unterschiedlicher Augenzahlen 3:}\\
$P_Y(\{(a,b,c)\}) = \frac{1}{|\Omega|} \cdot 1 \cdot 3! \cdot = \frac{6}{216}, \forall a,b,c \in [1:6]$\\
\\

Erklärung der Formeln:\\
1. Faktor, $\frac{1}{|\Omega|}$ ist klar,\\
2. Faktor ist die Anzahl  möglicher Kombinationen von a bzw. a,b bzw. a,b,c.:\\
 Nur Permutation (x, x, x) wird auf (a, a, a) abgebildet, daher 1.\\
 Nur Permutationen (x, y, y), (y, x, y), (y, y, x) werden auf (a, b, b) abgebildet, daher 3.\\
 Das Selbe bei (a, a, b).\\
 Nur Permutation (x, y, z) wird auf (a, b, c) abgebildet, daher 1.\\
3. Faktor ist die Anzahl Möglichkeiten die Variablen untereinander zu ersetzen (entspricht Permutation der Anzahl der Variablen)\\
4. Faktor bei 2 Augenzahlen ist $\frac{1}{2}$ weil die Hälfte der Elemente der Form (x, y, y), (y, x, y), (y, y, x) auf je eins der beiden Möglichkeiten fällt wegen Gleichverteilung.\\
 

Der Rest von $P_Y$ ergibt sich aus $\sigma$-Additivität denn im obigen sind alle Ereignisse aus $\Omega'$ behandelt.\\\\
Wenn wir die Summe der Verteilung disjunkter Ereignisse aus $\Omega'$ bilden, kommt auch wie erwartet 1 raus:\\
$\sum_{\omega \in \Omega'} P_y(\{\omega\}) = \frac{1}{|\Omega|} \cdot (6\cdot 1 + \binom{6}{2} \cdot 3 + \binom{6}{2} \cdot 3 + \binom{6}{3} \cdot 6) = 1$



\end{enumerate}


\newpage
\section*{Aufgabe 3}
\begin{enumerate}
    \item 
    \item Dichtefunktion:
    \begin{equation}
        F^{'}(X) = \frac{1}{\rho(F^{-1}(x))} \notag 
    \end{equation}\\
    $F(x) =  \left \{
          \begin{aligned}
           &0, x < 0 \\
           &1, x \in [0,1]\\
           &1, x > 1
          \end{aligned}\right.$  \notag
\end{enumerate}

\newpage
\section*{Aufgabe 4}
\begin{enumerate}[a)]
 \item
 
 Die Würfel können wir hier offensichtlich wie Münzen behandeln, nur mit Wurfergebnis 1 und 2 statt 0 und 1.\\
 
 $X_1:{1,2}^2 \rightarrow [2:4]$\\
 $(w_1, w_2) \mapsto w_1+w_2$\\\\
 
 $X_1(1,1)=2\\
  X_1(1,2)=3\\
  X_1(2,1)=3\\
  X_1(2,2)=4$\\
 
  $X_2:{1,2}^2 \rightarrow [1:4]$\\
 $(w_1, w_2) \mapsto w_1\cdot w_2$\\\\
 
  $X_2(1,1)=1\\
  X_2(1,2)=2\\
  X_2(2,1)=2\\
  X_2(2,2)=4$\\
 
 $P_X: [2:4] \times [1:4] \rightarrow [0:1]$:\\
\begin{tabular}{|l|ccc|}
 \hline
\diagbox{X1}{X2} & 1 & 2 & 4\\
\hline
2 & $\frac{1}{4}$ & 0 & 0\\
3 & 0 & $\frac{1}{2}$ & 0\\
4 & 0 & 0 & $\frac{1}{4}$\\
\hline
 \end{tabular}
 
\item
\begin{tabular}{|l|cc|l|}
 \hline
 \diagbox{y1}{y2} & 1 & 2 & Sum \\
 \hline
 1 & 0.2 & 0.45 & 0.65\\
 2 & 0.3 & 0.05 & 0.35 \\
 \hline
 Sum & 0.5 & 0.5 & 1\\
 \hline
\end{tabular}

\end{enumerate}



\end{document}

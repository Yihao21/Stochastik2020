% header
\documentclass[10pt,a4paper]{article}

\usepackage[utf8]{inputenc}
\usepackage{hyperref}
\usepackage{enumerate}
\usepackage{amssymb}
\usepackage{amsmath}
\usepackage{amsthm}
\usepackage{colortbl}
\usepackage{diagbox}
\usepackage{ngerman}

% the document
\begin{document}

% create the title
% Please replace the data in brackets [] with actual data.
\title{Abgabe - Übungsblatt [$4$]\\
\small{Angewandte Mathematik: Stochastik}}
\author{ [Vincent Schönbach] \and [Yihao Wang]}
\date{\today}
\maketitle

\section*{Aufgabe 1}

\newpage
\section*{Aufgabe 2}
\begin{enumerate}[a)]
 \item 
$\Omega_3 = [1:6]^3 = \{(\omega_1, \omega_2, \omega_3) \mid \forall i: \omega_i \in [1:6]\}$\\
$A = 2 ^ \Omega$\\
$\Omega^\prime = \{(\omega_1, \omega_2, \omega_3) \mid \omega_1 < \omega_2 < \omega_3 \}$\\
$A^\prime  = 2 ^ {\Omega^\prime} $\\
\item

\textbf{$\#$unterschiedlicher Augenzahlen 0:} d.h. 3 Augenzahlen sind identisch. Es gibt offensichtlich insgesmat 6 Fälle.\\
\textbf{$\#$unterschiedlicher Augenzahlen 1:} d.h. 2 Augenzahlen sind identisch. Es gibt insgesmat 
$\binom{6}{2} \times 2 = $ 30 Fälle.\\
\textbf{$\#$unterschiedlicher Augenzahlen 2:} d.h. 3 Augenzahlen unterscheiden sich. Es gibt insgesmat 
$\binom{6}{3} = $ 20 Fälle.
\end{enumerate}


\newpage
\section*{Aufgabe 3}

\newpage
\section*{Aufgabe 4}
\begin{enumerate}[a)]
 \item
\begin{tabular}{|l|ccc|}
 \hline
\diagbox{X1}{X2} & 1 & 2 & 4\\
\hline
2 & $\frac{1}{4}$ & 0 & 0\\
3 & 0 & $\frac{1}{2}$ & 0\\
4 & 0 & 0 & $\frac{1}{4}$\\
\hline
 \end{tabular}
 
\item
\begin{tabular}{|l|cc|l|}
 \hline
 \diagbox{y1}{y2} & 1 & 2 & Sum \\
 \hline
 1 & 0.2 & 0.45 & 0.65\\
 2 & 0.3 & 0.05 & 0.35 \\
 \hline
 Sum & 0.5 & 0.5 & 1\\
 \hline
\end{tabular}

\end{enumerate}



\end{document}

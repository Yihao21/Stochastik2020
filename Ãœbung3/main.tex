% header
\documentclass[10pt,a4paper]{article}

%\usepackage[latin1]{inputenc}
\usepackage{hyperref}
\usepackage{enumerate}
\usepackage{amssymb}
\usepackage{amsmath}
\usepackage{amsthm}
\usepackage{colortbl}
\usepackage{ngerman}

% the document
\begin{document}

% create the title
% Please replace the data in brackets [] with actual data.
\title{Abgabe - Übungsblatt [$3$]\\
\small{Angewandte Mathematik: Stochastik}}
\author{ [Vincent Schönbach] \and [Yihao Wang]}
\date{\today}
\maketitle

\section*{Aufgabe 1}
\begin{enumerate}[a)]
 \item Beh.: $\mathcal{A} = 2^\Omega$ ist die kleinste solche $\sigma$-Algebra.\\[1ex]
 Bew.: Es ist nach Voraussetzung $\mathcal{G} \subseteq \mathcal{A}$. \\
 Sei nun $M \in 2^\Omega$ beliebig. Wir zeigen, dass $M$ auch in $\mathcal{A}$ ist und somit $\mathcal{A} = 2^\Omega$ gilt.\\
 1. Fall: $M = \emptyset$ oder $M = \Omega$ \\
   Nach $\sigma$-Algebra-Def. ist $M \in \mathcal{A}$. \\
 2. Fall: $M = \{a_1,...,a_n\}$ mit $a_1,...,a_n \in \Omega$. \\
 Wir sehen, dass $M= \{a_1\} \cup ... \cup \{a_n\}$ und da $\{a_1\}, ..., \{a_n\} \in \mathcal{G}\subseteq \mathcal{A}$,
 muss auch die Vereinigung, $M$, in $\mathcal{A}$ sein (nach $\sigma$-Algebra-Def.(c)).
 
  \qed
  
 \item
 Wahrscheinlichkeitsmaß $P : \mathcal{A} \rightarrow [0,1]$ mit $P(\{\omega\})=p(\omega)$ $\forall \omega\in\Omega.$ \\
 Für $\{a_1,...,a_n\}\in\mathcal{A}$ muss nach $\sigma$-Additivität gelten: \\
 $P(\{a_1,...,a_n\})=P(\{a_1\} \cup ... \cup \{a_n\}) = p(a_1) + ... + p(a_n)$\\
 Außerdem muss gelten $P(\Omega) = 1$ und somit $P(\emptyset) = 0$.\\
 Eindeutigkeit folgt aus Vorgabe und Anwendung der Definition vom Wahrscheinlichkeitsmaß.
 
 \item W-Raum $(\Omega, \mathcal{A}, P)$ mit \\
 $\Omega = [1:6]$,\\
 $\mathcal{A} = 2^\Omega$,\\
 $P:\mathcal{A}\rightarrow [0,1]$ mit \\
 $P(\{i\})=p(i)$ $\forall i\in[1:6]$, Rest von P ergibt sich wieder aus $\sigma$-Additivität\\
 $P(\{1,2\}) = \dfrac{3}{10} + \dfrac{3}{10} = \dfrac{6}{10}$ \\
 $P(\{1,6\}) = \dfrac{3}{10} + \dfrac{1}{10} = \dfrac{4}{10}$ \\
 
\end{enumerate}

\newpage

\section*{Aufgabe 2}
Da $(\Omega, \mathcal{A}, P_1)$ W-Raum ist, ist $(\Omega, \mathcal{A})$ Messraum.\\
Bleibt zu zeigen, dass $P:= \alpha \cdot P_1 + (1-\alpha)\cdot P_2$ W-Maß ist.\\[1ex]
1. Es muss $P(\Omega)=1$ gelten.

\begin{equation}
\begin{split}
P(\Omega) & = \alpha \cdot P_1(\Omega) + (1-\alpha)\cdot P_2(\Omega) \\
    &\overset{\mathrm{P_1, P_2 \text{W-Maß}}}{=}\alpha \cdot 1 + (1-\alpha) \cdot 1 \\
    & = \alpha + 1-\alpha \\
    & = 1
\end{split}
\end{equation}\\
2. Es muss die $\sigma$-Additivität gelten.\\
Seien $A_1,... \in \mathcal{A}$ paarweise disjunkte Ereignisse.\\

\begin{equation}
\begin{split}
P(\bigsqcup \limits_{i \geq 1 } A_i ) & = \alpha \cdot P_1(\bigsqcup \limits_{i \geq 1 } A_i ) + (1-\alpha)\cdot P_2(\bigsqcup \limits_{i \geq 1 } A_i) \\
&\overset{\mathrm{P_1, P_2 \text{W-Maß}}}{=} \alpha \cdot \sum \limits_{i \geq 1 } P_1(A_i) + (1-\alpha) \cdot \sum \limits_{i \geq 1 } P_2(A_i) \\
& = \sum \limits_{i \geq 1 }(\alpha \cdot P_1(A_i) + (1-\alpha) \cdot P_2(A_i)) \\
& = \sum \limits_{i \geq 1 } P(A_i)
\end{split}
\end{equation}\\
\qed
\newpage

\section*{Aufgabe 3}
\begin{enumerate}[a)]
\item
Idee: Zeigen, dass es eine Injektion von $\{0,1\}^\mathbb{N}$ nach $C$ gibt?

\item
Man kann sehen, dass $A_n$ aus $2^n$ Intervallen der Größe $(\frac{1}{3})^n$ besteht.\\
$\Rightarrow \lambda ^1(A_n) = 2^n \cdot (\frac{1}{3})^n = (\frac{2}{3})^n < 1$\\
$\Rightarrow \lim \limits_{n\rightarrow \infty}(\lambda^1(A_n)) = 0$

Nun ist $0 \leq \lambda^1( \bigcap \limits_{n=0}^\infty A_n) \leq \lim \limits_{n\rightarrow \infty}(\lambda^1(A_n)) = 0$\\
Und daher ist $\lambda^1(C) = \lambda^1( \bigcap \limits_{n=0}^\infty A_n) = 0$.

\end{enumerate}
\newpage

\section*{Aufgabe 4}
Die Aufgabenstellung ist nicht so ganz klar wie es gemeint ist (insb. den Ereignisraum).
Wir haben jetzt nicht berücksichtigt, wie die drei Personen nebeneinandersitzen (ob abc, acb, bca,...)
und auch nicht wie die anderen 5 Plätze belegt sind.
\begin{enumerate}[a)]
\item
$\Omega = \{(w_1,...,w_8) \mid w_i \in \{0,1\} \land \sum \limits_{i=1}^8 w_i = 3\}$\\
$\mathcal{A} = 2^\Omega$\\
$P(\Omega)=1$ und $\forall w \in \Omega: P(\omega) = \frac{1}{|\Omega|}$ \\
(Rest von $P$ ergibt sich aus $\sigma$-Additivität.)
\item
$A =  \{(w_1,...,w_8) \in \Omega \mid w_i = w_{i+1} = w_{i+2} = 1, i \in [1:6] \lor w_7 = w_8 = w_1 = 1 \lor w_8 = w_1 = w_2 = 1\}$\\
\item
$|\Omega| = \binom{8}{3} = 56$\\
$|A|=8$ \\
$P(A) = \frac{8}{56} = \frac{1}{7}$
\end{enumerate}
\end{document}

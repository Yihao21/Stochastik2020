% header
\documentclass[10pt,a4paper]{article}

\usepackage[latin1]{inputenc}
\usepackage{hyperref}
\usepackage{amssymb}
\usepackage{enumerate}
\usepackage{amsmath}     
\usepackage{ngerman}

% the document
\begin{document}

% create the title
% Please replace the data in brackets [] with actual data.
\title{Abgabe - �bungsblatt [$1$]\\
\small{Angewandte Mathematik: Stochastik}}
\author{ [Vincent Sch�nbach] \and [Yihao Wang]}
\date{\today}
\maketitle

\section*{Aufgabe 1}
\begin{enumerate}[a)]
\item Immer falsch. $w$ ist Tupel und $A$ ist Menge.
\item Immer richtig. Menge selber geh�rt zu seiner eigenen Potenzmenge.
\item Immer richtig. $w \in \Omega \Rightarrow \{ w \} \in A$.
\item Immer richtig. $w$ ist Tupel.
\item Immer falsch. $w$ ist Tupel, nicht Menge.
\item Immer richtig. Da $w \in \Omega$.
\item Im Allgemeinen falsch.
\item Immer richtig. A ist die Potenzmenge von $\Omega$. Alle diese Elemente sind drin.
\end{enumerate}

\section*{Aufgabe 2}


\section*{Aufgabe 3}
\begin{enumerate}[a)]
 \item 
 \item Die Reihenfolge von Kombination eindeutig: 
 \[
 \frac{1}{49 \times 48 \times \ldots \times 44}
 \]
 
 Die Reihenfolge ist egal:
  \[
 \frac{1}{\binom{49}{6}}
 \]
\end{enumerate}

\section*{Aufgabe 4}
\begin{enumerate}[a)]
 \item 
 Z�hler: Die Wahrscheinlichkeit, die k-mal Kopf von n-mal zu erhalten.
 
 Nenner: Alle Ergebnisse $ \Rightarrow $ n-mal fair werfen $({\frac 12})^{n}$
\item
\[
 \frac{\binom{n}{1}}{2^{n}} = \frac{n}{2^{n}}
\]

 \end{enumerate}

\end{document}

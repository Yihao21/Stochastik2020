% header
\documentclass[10pt,a4paper]{article}

\usepackage[latin1]{inputenc}
\usepackage{hyperref}
\usepackage{enumerate}
\usepackage{amssymb}
\usepackage{amsmath}
\usepackage{amsthm}
\usepackage{colortbl}
\usepackage{ngerman}

% the document
\begin{document}

% create the title
% Please replace the data in brackets [] with actual data.
\title{Abgabe - �bungsblatt [$1$]\\
\small{Angewandte Mathematik: Stochastik}}
\author{ [Vincent Sch�nbach] \and [Yihao Wang]}
\date{\today}
\maketitle

\section*{Aufgabe 1}
\begin{enumerate}[a)]
 \item 
 \begin{tabular}{|c|c|c|c|}
 \hline
  Geweinn & 1 & -9 & -1\\
  \hline
  P & $\frac{25}{26}$ & $\frac{1}{36}$ & $\frac{10}{36}$\\
  \hline
 \end{tabular}
 
 $E_p(Gewinn) = 1 \times \frac{25}{36} - 9  \times \frac{1}{36} - 1 \times \frac{10}{36} = \frac{1}{6}$
 \item  Wahrscheinlichkeitsverteilung:
 \begin{tabular}{|*{7}{c|}}
  \hline
  X & 1 & 2 & 3 & 4 & 5 & 6\\
  \hline
  P & $\frac{1}{2}$ & $\frac{1}{4}$ & $\frac{1}{8}$ & $\frac{1}{16}$ & $\frac{1}{32}$ & $\frac{1}{32}$ \\
  \hline
 \end{tabular}
 
 Bei 6 kann das Wappen entweder erscheint oder nicht erscheint, deswegen $\frac{1}{64} \times 2$.\\
 $E_p(X) = 1 \times \frac{1}{2} + 2 \times \frac{1}{4} + 3 \times \frac{1}{8} + 4 \times \frac{1}{16} + 5 \times \frac{1}{32} + 6 \times \frac{1}{32} = 1.96875$
\end{enumerate}


\newpage
\section*{Aufgabe 2}
\begin{align}
E_p(X) &= \sum_{k=0}^{\infty} k \cdot \frac{e^{-\lambda} \cdot \lambda^k}{k!}\\
&= \sum_{k=1}^{\infty} k \cdot \frac{e^{-\lambda} \cdot \lambda^k}{k!}\\ 
&= \sum_{k=1}^{\infty} \frac{e^{-\lambda} \cdot \lambda^k}{(k-1)!}\\
&= \lambda \cdot \sum_{k=1}^{\infty}\frac{e^{-\lambda} \cdot \lambda^{k-1}}{(k-1)!}\\
&= \lambda
\end{align}
\qed

Erkl�rung:
\begin{enumerate}[(1)]
 \item Nach Formel $E_p(X) = \sum_{\omega \in \Omega} X(\omega)P(\omega)$
 \item F�r k=0, ist das Ergebnis 0, wir k�nnen direkt mit 1 anfangen.
 \item kurze Umformung.
 \item wir nehmen ein $\lambda$ raus. Und wir haben hier wieder eine Posionsverteilung.
 \item Die Summe von Posion-Verteilung ist 1.
\end{enumerate}

\begin{align*}
 E_p(X^2) &= \sum_{k=0}^{\infty} k^2 \cdot \frac{e^{-\lambda} \cdot \lambda^k}{k!}\\
 &= \lambda e^{-\lambda} \sum_{k=1}^{\infty} \frac{k \cdot \lambda^{k-1}}{(k-1)!}\\
 &= \lambda e^{-\lambda} \sum_{k=1}^{\infty} \frac{(k-1+1) \cdot \lambda^{k-1}}{(k-1)!}\\
 &=\lambda e^{-\lambda} \sum_{k=1}^{\infty} \frac{(k-1) \cdot \lambda^{k-1}}{(k-1)!} + \frac{\lambda^{k-1}}{(k-1)!}\\
 &=\lambda e^{-\lambda} \cdot(\sum_{k=1}^\infty \frac{(k-1) \cdot \lambda^{k-1}}{(k-1)!} + \sum_{k=1}^\infty \frac{\lambda^{k-1}}{(k-1)!})\\
 &=\lambda e^{-\lambda} \cdot(\lambda \cdot \sum_{k=1}^\infty \frac{\lambda^{k-2}}{(k-2)!} + \sum_{i = 0}^\infty \frac{\lambda^{i}}{i!}) \qquad i=k-1\\
&=\lambda e^{-\lambda} \cdot(\lambda \cdot \sum_{j=0}^\infty \frac{\lambda^{j}}{j!} + \sum_{i = 0}^\infty \frac{\lambda^{i}}{i!}) \qquad j=k-2\\
&= \lambda(\lambda + 1)
\end{align*}

Erkl�rung:
Tja, wir glauben der Beweis schon ausf�hrlich genug ist und erkl�rt selbst. Nur k-1+1 ist bisschen Tricky in den Beweis. Nun benutzen wir den Hinweis der Aufgabe.
\begin{align*}
 V_p(X) &= E_p(X^2) - E_p(X)^2\\
 &= \lambda(\lambda + 1) - \lambda ^2 = \lambda\\
 \end{align*}
\qed


\newpage
\section*{Aufgabe 3}
\begin{align}
 E_p(X) &= \sum_{n=0}^\infty n \cdot p \cdot (1-p)^{n-1} \notag \\
 &= p \cdot \sum_{n=1}^\infty n \cdot (1-p)^{n-1} \notag \\
 &= p \cdot \frac{1}{(1-(1-p))^2}\\
 &= \frac{1}{p} \notag
 \end{align}\qed
 
 Erkl�rung: Die haupte Idee ist nicht so unterschied wie A2, wir benutzen die Formel von Erwartungswert direkt. Und nat�rlich den Hinweis der Aufgabe.
 
\begin{align*}
 E_p(X^2) &= \sum_{n=0}^\infty n^2 \cdot p \cdot (1-p)^{n-1} \\
&= p \cdot \sum_{n=1}^\infty n^2 \cdot (1-p)^{n-1}\\
&= p \cdot \sum_ {n=1}^\infty (n+1-1)\cdot n \cdot (1-p)^{n-1}\\
&=  p \cdot \sum_ {n=1}^\infty (n+1)\cdot n \cdot (1-p)^{n-1} - \sum_{n=1}^\infty n \cdot (1-p)^{n-1}\\
&= p \cdot (\frac{2}{(1-(1-p))^3} - \frac{1}{(1-(1-p))^2})\\
&= \frac{2}{p^2} - \frac{1}{p}
\end{align*}

\begin{align*}
 V_p(X) &= E_p(X^2) - E_p(X)^2\\
 &= \frac{2}{p^2} - \frac{1}{p} - \frac{1}{p^2}\\
 &= \frac{1-p}{p^2}
\end{align*}\qed

\newpage
\section*{Aufgabe 4}
\begin{enumerate}[a)]
 \item  Die ZV, die die Zahl der W�rfe bis die i-te verschiedene Zahl geworfen ist, passt genau die geometrische Verteilung.\\
 $X_1 = 1$\\
 $P(X=i) = (\frac{5}{6}) \times (1-\frac{5}{6})^{(i-1)}$\\
 Die Erwartungswert von geometrische-Verteilung ist $\frac{1}{p}$(von A3). Wir zeigen hier mit einem Beispiel, f�r die 2-te verschiedene Zahl, die Wahrscheinlichkeit davon ist:\\
 $P(n) = \frac{5}{6} \times (1-\frac{5}{6})^{n-1} \Rightarrow E_p(2) = \frac{1}{p} = \frac{6}{5}$\\
 sonst ist analog, der Erwartungswert folgt:\\
 $1 + \frac{6}{5} + \frac{6}{4} + \ldots + 6 = 14.7$
\item direkt das Ergebnis von A3 nutzen\\
$V(X) = \frac{1-\frac{4}{6}}{\frac{4}{6}^2} = \frac{3}{4} = 0.75$
\end{enumerate}



\newpage
Ps: Hi Matthias. Ich(Yihao) muss noch mal vorstellen. Wir glauben, dass dieses Blatt haben wir gut gemacht. Schau mal, welche ich vorstellen soll. danke!



\end{document}

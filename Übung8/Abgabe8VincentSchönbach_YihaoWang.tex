% header
\documentclass[10pt,a4paper]{article}

\usepackage[latin1]{inputenc}
\usepackage{hyperref}
\usepackage{enumerate}
\usepackage{amssymb}
\usepackage{amsmath}
\usepackage{amsthm}
\usepackage{colortbl}
\usepackage{ngerman}

% the document
\begin{document}

% create the title
% Please replace the data in brackets [] with actual data.
\title{Abgabe - �bungsblatt [$8$]\\
\small{Angewandte Mathematik: Stochastik}}
\author{ [Vincent Sch�nbach] \and [Yihao Wang]}
\date{\today}
\maketitle

\section*{Aufgabe 1 Unabh�nigkeit $\&$ Normalverteilung}
Wenn die ZV X und Y unabh�ngig sind, z.zg: $P(X,Y) = P(X) \cdot P(Y)$
\begin{align}
 P(X,Y) &= \frac{1}{2\pi}\int \int_{X \times Y} exp(-\frac{x^2 + y^2}{2})dxdy \notag \\
 &= \frac{1}{2\pi}\int_{-\infty}^{x_0} \left[ \int_{-\infty}^{y_0} exp(-\frac{x^2 + y^2}{2}) dy\right]dx\\
 &= \frac{1}{2\pi}\int_{-\infty}^{x_0} \left[ \int_{-\infty}^{y_0} exp(-\frac{x^2}{2}) \cdot exp(-\frac{y^2}{2}) dy\right]dx \\
&= \frac{1}{2\pi}\int_{-\infty}^{x_0} exp(-\frac{x^2}{2}) dx\cdot \int_{-\infty}^{y_0} exp(-\frac{y^2}{2}) dy\notag \\
&= \frac{1}{2\pi}\int \int_{X} exp(-\frac{x^2 + y^2}{2})dxdy \cdot \frac{1}{2\pi}\int \int_{Y} exp(-\frac{x^2 + y^2}{2})dxdy \notag\\
&= P(X) \cdot P(Y) \notag
\end{align}
\qed \\
Erkl�rung:\\
(1) Wir benutzen Satz von Fubini hier. $x_0 \in X$ und $y_0 \in Y$. Da die beide ZV sind, der untere Schrank ist negativ unendlich.\\
(2) Eigenschaft von Integral.



\newpage
\section*{Aufgabe 2 Unabh�nigkeitskriterium}
``$\Rightarrow$'':$(Y_i)_{i \in \left[1:n\right]}$ ist unabh�ngig.\\
d.h. $P(Y_1 = \omega_1, \ldots, Y_n = \omega_n) = P(Y_1) \cdot \ldots \cdot P(Y_n) = \prod_{i=1}^nP(Y_i = \omega_i)$\\
``$\Leftarrow$'' analog.


\newpage
\section*{Aufgabe 3 Tutorium}
\begin{enumerate}[a)]
 \item Bernoulli-Verteilung.
 \item $\Omega \rightarrow \left[0:n\right]$ \\
       A: $2^\Omega$\\
       P: $A \rightarrow \left[0,1\right]$\\
       P = $\frac{\lvert A \rvert}{\Omega}$
 \item $\hat P=\sum_{i=k+1}^n i^p \cdot (n-i)^{1-p}$
\end{enumerate}

\newpage
\section*{Bonusaufgabe 4}
\begin{enumerate}[a)]
 \item Geometrische Verteilung
 \item $\hat \Omega = \lvert\{(\omega_1,\ldots,\omega_n) \mid \omega_1 = \ldots = \omega_{n-1} = 1, \omega_n = 0\} \rvert$ 1:stattgefunden 0:ausgef�llt.\\
 $\hat A = 2^{\hat \Omega}$
 \item $\hat P$ stattgefunden werden.\\
 P(Aufgabe4.3) = $(\hat P)^6 \cdot (1-\hat P)^7 + (\hat P)^7 \cdot (1-\hat P)^8 $
\end{enumerate}


\end{document}

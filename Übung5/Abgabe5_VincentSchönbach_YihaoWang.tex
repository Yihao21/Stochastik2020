% header
\documentclass[10pt,a4paper]{article}

\usepackage[utf8]{inputenc}
\usepackage{hyperref}
\usepackage{enumerate}
\usepackage{amssymb}
\usepackage{amsmath}
\usepackage{amsthm}
\usepackage{colortbl}
\usepackage{ngerman}

% the document
\begin{document}

% create the title
% Please replace the data in brackets [] with actual data.
\title{Abgabe - Übungsblatt [$5$]\\
\small{Angewandte Mathematik: Stochastik}}
\author{ [Vincent Schönbach] \and [Yihao Wang]}
\date{\today}
\maketitle

\section*{Aufgabe 1}
\begin{enumerate}[a)]
 \item $\Omega_n = [1:m]^n = \{(\omega_1, \ldots, \omega_n) \mid \forall i: \omega_i \in [1:m]\}, m, n \in \mathbb{N}$\\
 $\mathcal{A} = 2^\Omega$\\
 $P:\mathcal{A} \rightarrow [0,1]$\\
 mit $P(\{w\}) = \frac{1}{m^n},\forall w\in\Omega$\\
 $P(A) = \sum_{w\in A}P(\{w\}),\forall A \in \mathcal{A}$\\
 \item 
 $X:\Omega \rightarrow [0:n]$ mit\\
 $X(w) = X((w_1,...,w_n)) = |\{i\in[0:n-1] \mid w_i = w_{i+1}\}|$
 \item
 $P_X(\{k\}) := P(X^{-1}(\{k\}))\\
 =\frac{|\{w\in\Omega : X(w)=k\}}{|\Omega|}$
 
 $m=2, n=3, k=1$:\\
 Wir haben zwei unterscheidbare Kugeln und nennen die r(rot) $\&$ b(blau).
 Es gibt insgesamt $2^3$ Möglichkeiten: bbb/\textbf{bbr}/brb/\textbf{rbb}/\textbf{brr}/rbr/\textbf{rrb}/rrr\\
 Also \\ $X^{-1}(\{1\}) = \{(\omega_a,\omega_b,\omega_c) \mid \omega_a = \omega_b \neq \omega_c \lor  \omega_a = \omega_c \neq \omega_b \lor  \omega_c = \omega_b \neq \omega_a)\}\\
 = \{(1, 1, 2), (1,2,2), (2,2,1), (2,1,1)\}$\\
 Offensichtlich ist $P(X=k) = \frac{|A|}{m^n} = \frac{4}{8} = \frac{1}{2}$\\
 
\end{enumerate}


\newpage
\section*{Aufgabe 2}
\begin{enumerate}[a)]
 \item 
 W-Raum $(\Omega, \mathcal{A}, P)$ mit \\
 $\Omega = [1:4]^{18}$,\\
 $\mathcal{A} = 2^\Omega$,\\
 
 
 Wenn ich die Prüfung von 18 Aufgaben bestehen möchte, muss ich zu mindestens 11 Aufgaben richtig beantworten.\\
 Wir schauen mal zunächst auf die Möglichkeit, dass ich genau 11 Aufgaben richtig beantwortet habe. Da passt genau die Binomial-Verteilung\\
 $\binom{18}{11} \times (\frac{3}{4})^7 \times (\frac{1}{4})^{11}$\\
 Mindestens 11 Aufgaben richtig ergibt sich dann als:\\
 $P(bestanden) = P(\{w\}) = \sum_{i=11}^{18} \binom{18}{i} \times (\frac{3}{4})^{18-i} \times (\frac{1}{4})^i \approx 1.24 \times 10^{-3} $ \\
 $\forall w\in \Omega$\\
  $P(A) = \sum_{w\in A}P(\{w\}),\forall A \in \mathcal{A}$\\
 
 \item
 die Intuition ist gleich:\\
  $\Omega = [1:3]^{18}$\\
 $\mathcal{A} = 2^\Omega$\\
 $P(bestanden) = P(\{w\})= \sum_{i=11}^{18} \binom{18}{i} \times (\frac{2}{3})^{18-i} \times (\frac{1}{3})^i \approx 0.0144$ \\
  $\forall w\in \Omega$\\
  $P(A) = \sum_{w\in A}P(\{w\}),\forall A \in \mathcal{A}$\\

 
\item
$\Omega = [1:2]^{18}$\\
$\mathcal{A} = 2^\Omega$\\
$P(bestanden) = P(\{w\})=\sum_{i=11}^{18} \binom{18}{i} \times (\frac{1}{2})^{18-i} \times (\frac{1}{2})^i \approx 0.2403$ \\
  $\forall w\in \Omega$\\
  $P(A) = \sum_{w\in A}P(\{w\}),\forall A \in \mathcal{A}$\\

\end{enumerate}


\newpage
\section*{Aufgabe 3}
\begin{enumerate}[a)]
 \item 
 $\Omega = [0:1]^{12}$\\
$\mathcal{A} = 2^\Omega$\\
 $P(bestanden) = P(\{w\})= $\sum_{i=8}^{12} \binom{12}{i} \times (\frac{1}{2})^{12-i} \times (\frac{1}{2})^i \approx 0.1938$\\
   $\forall w\in \Omega$\\
  $P(A) = \sum_{w\in A}P(\{w\}),\forall A \in \mathcal{A}$\\\\
 Wir finden, dass die Binomial-Verteilung schon reicht, aber natürlich kann man es mit Poisson-Verteilung approximieren.
 
 \item
  Wir übersetzen die Aufgabestellung, man muss nun mindestens 6 Aufgaben von 10 Fragen richtig antworten.\\
  $\Omega = [0:1]^{10}$\\
$\mathcal{A} = 2^\Omega$\\
  $P(bestanden) = P(\{w\})= \sum_{i=6}^{10} \binom{10}{i} \times (\frac{1}{2})^{10-i} \times (\frac{1}{2})^i \approx 0.3769$\\
    $\forall w\in \Omega$\\
  $P(A) = \sum_{w\in A}P(\{w\}),\forall A \in \mathcal{A}$\\\\
 wie a), Binomial-Verteilung.
 
 \item 
   $\Omega = \{(w_1,...w_{12}): w_i\in\{0,1\} \land \sum_{i=1}^{12}w_i=6\}$\\
$\mathcal{A} = 2^\Omega$\\
$P(bestanden) = P(\{w\})= \sum_{i=8}^{12}\sum_{i=j-6}^{6}\frac{\binom{6}{i}\cdot\binom{6}{j-i}}{\binom{12}{6}}$
    $\forall w\in \Omega$\\
  $P(A) = \sum_{w\in A}P(\{w\}),\forall A \in \mathcal{A}$\\\\
  
Hypergeometrische Verteilung.\\
Wir ziehen zufällig $n=6$ mal aus $N=12$ Möglichkeiten ohne Zurücklegen und ohne Berücksichtigung der Reihenfolge. $N_{ja}=6$, $N_{nein} = 6$. Gesucht ist die Wahrscheinlichkeit, mind. 8 Richtige zu ziehen. Daher betrachten wir die Fälle, genau 8, 9, 10, 11, 12 Richtige zu ziehen einzeln und summieren darüber auf. (äußere Summe)\\
Bei z.B. genau 8 richtigen Fällen bedeutet es, entweder 2 von den ja-Antworten und 6 von den nein-Antworten richtig zu beantworten oder 3 von den ja-Antworten und 5 von den nein-Antworten oder usw. ... oder 6 von den ja-Antworten und 2 von den nein-Antworten richtig zu beantworten. Dadurch ergibt sich die innere Summe.\\
Das auszurechnen ist zu viel Arbeit, wenn eine größere W-Keit rauskommt ist es besser als 1.
 
\end{enumerate}


\newpage
\section*{Aufgabe 4}
\begin{enumerate}[a)]
 \item 
 $\Omega = [K,G,B]^{24}$\\
 $\mathcal{A} = 2 ^ \Omega$\\
 $P(\{w\}) = P(\{(w_1,...,w_{24})\}) = (\frac{3}{10})^k \cdot (\frac{1}{10})^g \cdot (\frac{6}{10})^b$\\
 mit $k=|\{i\in [1:24] : w_i=K\}|,\\
 g=|\{i\in [1:24] : w_i=G\}|,\\
b=|\{i\in [1:24] : w_i=B\}|$\\
$\forall w\in\Omega$
 \item
 $P(\{(w_1,...,w_{24})\in\Omega : |\{i\in [1:24]: w_i=K\}|=5 \\\land |\{i\in [1:24]: w_i=G\}|=3\})\\
 =\binom{24}{5} \times (\frac{3}{10})^5 \times \binom{19}{3} \times (\frac{1}{10})^3 \times \binom{16}{16} \times (\frac{6}{10})^{16} \approx 0.02823$
\end{enumerate}



\end{document}

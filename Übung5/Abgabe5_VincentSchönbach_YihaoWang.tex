% header
\documentclass[10pt,a4paper]{article}

\usepackage[latin1]{inputenc}
\usepackage{hyperref}
\usepackage{enumerate}
\usepackage{amssymb}
\usepackage{amsmath}
\usepackage{amsthm}
\usepackage{colortbl}
\usepackage{ngerman}

% the document
\begin{document}

% create the title
% Please replace the data in brackets [] with actual data.
\title{Abgabe - �bungsblatt [$5$]\\
\small{Angewandte Mathematik: Stochastik}}
\author{ [Vincent Sch�nbach] \and [Yihao Wang]}
\date{\today}
\maketitle

\section*{Aufgabe 1}
\begin{enumerate}[a)]
 \item $\Omega_n = [1:m]^n = \{(\omega_1, \ldots, \omega_m) \mid \forall i: \omega_i \in [1:m]\}, m, n \in \mathbb{N}$\\
 $A = 2^\Omega$\\
 $P = \frac{1}{m^n}$
 \item 
 $X = \{\#(\omega_i,\omega_{i+1}) \mid \forall i \in [0:n-1]: \omega_i = \omega_{i+1})\}$
 \item
 Wir haben zwei unterscheidbare Kugeln und nennen die r(rot) $\&$ b(blau).
 Es gibt inegesamt $2^3$ M�glichkeiten: bbb/\textbf{bbr}/brb/\textbf{rbb}/\textbf{brr}/rbr/\textbf{rrb}/rrr\\
 Offensichtlich ist $P(X=k) = \frac{4}{8} = \frac{1}{2}$\\
 $X^{-1}(\{1\}) = \{(\omega_a,\omega_b,\omega_c) \mid \omega_a = \omega_b \neq \omega_c \lor  \omega_a = \omega_c \neq \omega_b \lor  \omega_c = \omega_b \neq \omega_a)\}$\\  
 
\end{enumerate}


\newpage
\section*{Aufgabe 2}
\begin{enumerate}[a)]
 \item 
 Wenn ich die Pr�fung bestehen m�chte, muss ich zum mindestens 11 Aufgaben richtig antworten.\\
 Wir schauen mal zun�chst die M�glichkeit, die ich genau 11 Aufgaben richtig geantwortet habe. Das passt genau die Binomial-Verteilung\\
 $\binom{18}{11} \times (\frac{3}{4})^7 \times (\frac{1}{4})^{11}$\\
 zum mindestens 11 Aufgaben bezeichnet man als:\\
 $P(bestanden) = \sum_{i=11}^{18} \binom{18}{i} \times (\frac{3}{4})^{18-i} \times (\frac{1}{4})^i \approx 1.24 \times 10^{-3} $ \\
 $\Omega = [1:4]^{18}$\\
 $\mathcal{A} = 2^\Omega$\\
 \item
 die Intuition ist gleich:\\
 $P(bestanden) = \sum_{i=11}^{18} \binom{18}{i} \times (\frac{2}{3})^{18-i} \times (\frac{1}{3})^i \approx 0.0144$ \\
 $\Omega = [1:3]^{18}$\\
 $\mathcal{A} = 2^\Omega$\\
\item
$P(bestanden) =\sum_{i=11}^{18} \binom{18}{i} \times (\frac{1}{2})^{18-i} \times (\frac{1}{2})^i \approx 0.2403$ \\
$\Omega = [1:2]^{18}$\\
$\mathcal{A} = 2^\Omega$\\
\end{enumerate}


\newpage
\section*{Aufgabe 3}
\begin{enumerate}[a)]
 \item 
 $\sum_{i=8}^{12} \binom{12}{i} \times (\frac{1}{2})^{12-i} \times (\frac{1}{2})^i \approx 0.1938$\\
 Wir finden, dass die Binomial-Verteilung schon reicht, aber nat�rlich kann man es mit Poisson-Verteilung approximiert werden.
 \item
 Wir �bersetzen die Aufgabestellung, man muss nun mindestens 6 Aufgaben von 10 Fragen richtig antworten.\\
 $\sum_{i=6}^{10} \binom{10}{i} \times (\frac{1}{2})^{10-i} \times (\frac{1}{2})^i \approx 0.3769$\\
 wie a), Binomial-Verteilung.
 \item 
 $\frac{1 + \binom{12}{2} + \binom{12}{8}}{\binom{12}{6}} \approx 0.608$\\
 Ja, es ist g�nstiger. 
\end{enumerate}


\newpage
\section*{Aufgabe 4}
\begin{enumerate}[a)]
 \item 
 1 : ein kleiner Gewinn, 2 : ein gr��erer Geweinn, 3 : Bild\\
 $\Omega = [1:3]^{24} = \{\#(\omega_i) \mid \omega_i \in [1:2]\}.$\\
 $\mathcal{A} = 2 ^ \Omega$\\
 \item
 $\binom{24}{5} \times (\frac{3}{10})^5 \times \binom{19}{3} \times (\frac{1}{10})^3 \times (\frac{6}{10})^{16} \approx 0.02823$
\end{enumerate}



\end{document}
